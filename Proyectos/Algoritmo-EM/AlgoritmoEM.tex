\documentclass[utf8,spanish,xcolor={table,dvipsnames},12pt]{beamer}
\uselanguage{spanish}
\languagepath{spanish}

\usepackage{dsfont}

\title[Algoritmo EM]{Algoritmo  EM}
\subtitle{T\'opicos de Investigaci\'on: Machine Learning}
\author{Luis Alberto Evangelista}
\institute[UNI]{Universidad Nacional de Ingeniería}
\date{05 de Octubre}

\usetheme{Berlin}
\useinnertheme{rectangles}
\useoutertheme{infolines}
%\usecolortheme{wolverine}
%\usefonttheme[onlymath]{serif} %%% SOLO MATEMATICA SE MANTIENE

\begin{document}
\begin{frame}
\maketitle
\end{frame}

\begin{frame}{Introducción}

 El algoritmo EM (Expectation-Maximization) es una técnica de optimización originalmente introducida por Dempster, Laird y Rubin (1997), en su publicación \textit{Maximun Likelihood from Imcomplete Data via the EM algorithm}. 
 
 \vspace{0.4cm}
 
 Se utiliza en estad\'istica para encontrar estimadores de verosimilitud(VM) de parámetros en modelos probabilísticos que dependen de variables no observables (datos pérdidos).
\end{frame}

\begin{frame}{Introducción}
 \begin{itemize}
   \item El algoritmo EM alterna pasos de esperanza (paso E),donde se calcula la esperanza de la verosimilitud mediante la inclusión de variables latentes como si fueran observables,   
   \item y un paso de Maximización (paso M),donde se calculan estimadores de VM de los parámetros mediante la maximización de la verosimilitud esperada del paso E.
       Los parámetros que se encuentra en el paso M se usan para comenzar el paso E siguiente, y así el proceso se repite.
 \end{itemize}

 \end{frame}


\begin{frame}{Preliminares: Desigualdad De Jensen}

Sea $f$ una función cuyo dominio es el conjunto de los números reales. Sabemos que $f$ es función convexa si $f^{''}(x)\geq 0$, para todo $x\in \mathds{R}$.\\

 En caso que $f$ sea  función real en varias variables , la función $f$ es convexa si su Hessiana $H$ es semidefinida positiva $(H\geq 0)$.\\

\vspace{0.2cm}


 Si $f^{''}(x)>0$ para todo $x\in \mathds{R}$ ( Hessiana $H$ de $f$ es definida positiva:$H>0$),decimos que f es estrictamente convexa.\\
 
 \vspace{0.2cm}
 
 La desigualdad de Jensen puede expresarse de la siguiente forma, escrito como el siguiente teorema
\end{frame}

\begin{frame}{Desigualdad de Jensen}
\begin{theorem}
  Sea $f$ una función y $X$ una variable aleatoria entonces
  $$f(E[X])\leq E[f(X)]$$
  
  \vspace{0.2cm}
  
  Además si $f$ es estrictamente convexa  entonces:
  
  \vspace{0.2cm}
  
  $E[f(X)]=f(E[X])$ si y s\'olo si $X=E[X]$ con probabilidad $1$ (i.e $X$ es constante).
\end{theorem}
\end{frame}

\begin{frame}
\begin{proof}
  \pause\
  Veamos en el caso que
  \pause\
  \begin{enumerate}
    \item X sea finito, en efecto:\\
          Sea $p_{i}=P\{X=x_{i}\}$ entonces $E[X]= \displaystyle \sum_{i=1}^{n}p_{i}x_{i}$ es una combinación convexa de valores de X.\\
          Como X es una función convexa entonces \\
          $f(E[X])=f(\displaystyle \sum_{i=1}^{n}p_{i}x_{i})\leq \sum_{i=1}^{n}p_{i}f(x_{i})=E[f(X)]$
    \pause\ 
    \item X sea numerable, en efecto:\\
     Como X toma una cantidad numerable de valores $x_{i}$ con probabilidad $p_{i}$.\\
     Definimos $s_{n}=\displaystyle\sum_{i=1}^{n}p_{i}$, para cada $n\in \mathds{N}$\\
     \end{enumerate}
     \end{proof}
     \end{frame}

     \begin{frame}{}
       Notamos que $\displaystyle\sum_{i=1}^{n}\frac{p_{i}}{s_{n}}x_{i}$ es una combinación convexa.


     Luego como $f$ es convexa entonces
     \[
      f(\displaystyle\sum_{i=1}^{n}\frac{p_{i}}{s_{n}}x_{i})\leq \displaystyle\sum_{i=1}^{n}\frac{p_{i}}{s_{i}}f(x_{i})
      \]
      
      \vspace{0.2cm}
      
      Luego tomando limite cuando $n\rightarrow \infty$,  $s_{n}\rightarrow 1$ y de la continuidad de $f$ tenemos:
      
      \vspace{0.2cm}
      
      $f(E[X])=f(\displaystyle\sum_{i=1}^{\infty}p_{i}x_{i})\leq \sum_{i=1}^{\infty}p_{i}f(x_{i})=E[f(X)]$
     \end{frame}

\begin{frame}

 {\large \textbf{Recordar}} :
 
 \vspace{0.3cm}
 
  $f$ es concava(estrictamente concava) si y s\'olo si $-f$ es convexa (estrictamente convexa) i.e. $f(x)\leq 0,H\leq 0 (f(x)< 0,H<0)$

\vspace{0.3cm}

 El teorema puede reescribirse en términos de la función c\'oncava de la
 siguente forma:


\vspace{0.3cm}
 
 \begin{theorem}
 	Sea $f$ una función c\'oncava y $X$ una variable aleatoria entonces
 	$$E[f(X)]\leq f(E[X])$$
 \end{theorem}
\end{frame}


\begin{frame}{Algoritmo EM}

 Se supone que $X=(X_{1},X_{2},\cdots,X_{n})$ es una variable aleatoria con distribución conjunta de $g(X\mid \theta)$ y se quiere calcular \\
 \begin{center}
   $ \widehat{\theta}$=argmáx$L(\theta\mid X)$
 \end{center}
 
 \vspace{0.2cm}
 
 Donde $L(\theta\mid X)=g(X\mid\theta)$.
 
 \vspace{0.2cm}
 
 Consideremos los datos completos W provenientes de una muestra aleatoria constituida por $W=(X,Z)$,donde W representa los datos completos,X los datos observados y Z datos pérdidos.
 
 \vspace{0.2cm}
 
 
 La distribución conjunta de W es
 \begin{center}
   $f(W\mid \theta)=f(X,Z\mid\theta)=k(Z\mid\theta,X)g(X\mid\theta)$.
 \end{center} 

 \end{frame}
 

\begin{frame}{Algoritmo EM}
	
 ¿Cómo calculamos $L^{c}(\theta\mid W)=L^{c}(\theta\mid X,Z)$ si no conocemos Z?
 
 \vspace{0.4cm}
 
 \textbf{Respuesta}: No conocemos Z de $L^{c}(\theta\mid X,Z)$, así que la supondremos como variable aleatoria y calculamos una media.
 
 \vspace{0.2cm}
 
Considerando $g(X\mid\theta)=\int_{Z}f(X,Z\mid\theta)dZ$,donde $(X,Z)\sim f(X,Z\mid\theta)$. Entonces la distribución condicional de los datos pérdidos Z,dado los datos observados X es

\vspace{0.2cm}

\begin{center}
  $k(Z\mid\theta,X)=\frac{f(X,Z\mid\theta)}{g(X\mid\theta)}$
\end{center}


\end{frame}


\begin{frame}{Algoritmo EM}

Además existe una relación entre la verosimilitud para los datos completos $L^{c}(\theta\mid X,Z)$ y la verosimilitud para los datos observados$L(\theta\mid X)$ dada por

\begin{center}
	$L^{c}(\theta\mid X,Z)=k(Z\mid\theta,X)L(\theta\mid X)$
\end{center}
Luego tomando logaritmo


\begin{center}
	$\log L^{c}(\theta\mid X,Z)=\log k(Z\mid\theta,X)+\log L(\theta\mid X)$
\end{center}

Es decir :

\vspace{0.2cm}

$\log g(X\mid\theta)=\log L(\theta\mid X)=\log L^{c}(\theta\mid X,Z)-\log k(Z\mid\theta,X) $\

\end{frame}

\begin{frame}{Algoritmo EM}

Para un valor de $\theta_{0}$,calculando la esperanza con respecto a $k(Z\mid\theta,X)$
y utilizando la desigualdad de Jensen, se tiene

\vspace{0.2cm}

\begin{itemize}
	\item $\log L(\theta\mid X)$: datos observados
	\item $E_{\theta_{0}}[\log L^{c}(\theta\mid X,Z)]$:datos completos
	\item $E_{\theta_{0}}[\log k(Z\mid\theta,X)]$ :datos pérdidos, que cumplen :
\end{itemize}

\vspace{0.2cm}

\begin{center}
	$\log L(\theta\mid X)=E_{\theta_{0}}[\log L^{c}(\theta\mid X,Z)]-E_{\theta_{0}}[\log k(Z\mid\theta,X)]$ 
\end{center}
\vspace{0.2cm}

Al maximizar $\log L(\theta\mid X)$ se debe ignorar el término asociado solo a los datos pérdidos.
\end{frame}





\begin{frame}{Iteraciones}

\begin{enumerate}
  \item El valor esperado del $\log$-verosimilitud se denota por
   \begin{center}
     $ Q(\theta\mid\theta_{0},X)=E_{\theta_{0}}[\log L^{c}(\theta\mid X,Z)]$.
   \end{center}
   
  \item El algoritmo EM comienza maximizando en cada iteración $Q(\theta\mid\theta_{0},X)$.
  \item Si $\hat{\theta}_{(1)}$=argmáx$Q(\theta\mid\theta_{0},X)$ entonces $\hat{\theta}_{(0)}\rightarrow \hat{\theta}_{(1)}$.
  \item Se tienen secuencias de estimadores $\{\hat{\theta}_{(j)}\}$, donde \\
        $ \hat{\theta}_{(j)}$=argmáx $Q(\theta\mid\hat{\theta}_{(j-1)},X)$   
  \item Este esquema iterativo, en cada paso contiene un cálculo de esperanza y maximización.
  
\end{enumerate}
\end{frame}

\begin{frame}{El Algoritmo}
\pause\
Se comienza con un valor inicial $\hat{\theta}_{(0)}$ fijado por el investigador.\\
Repita
\pause\
\begin{enumerate}
  \item Calcule(Paso E)
        \begin{center}
          $Q(\theta\mid\hat{\theta}_{(m)},X)=E_{\hat{\theta}_{(m)}}[\log L^{c}(\theta\mid X,Z)]$,
         \end{center}
        donde la esperanza es con respecto a $k(Z\mid\hat{\theta}_{(m)},X)$ \\
        y establecer $m=0$.
   \pause\ 
  \item Maximizar $Q(\theta\mid\hat{\theta}_{(m)},X)$ en $\theta$ y tomar (paso M)
        \begin{center}
          $\hat{\theta}_{(m+1)}$=argmáx$\{Q(\theta\mid\hat{\theta}_{(m)},X):\theta\}$
        \end{center}
         y establecer $ m=m+1$.\\
         Los parámetros que se encuentran en el paso M se usan para comenzar el paso E, y así el proceso se repite.Es decir se fija el punto $\hat{\theta}_{(m+1)}=\hat{\theta}_{(m)}$
\end{enumerate}
\end{frame}
\begin{frame}{Bibliografía}
\begin{thebibliography}{9}

\bibitem{Kop04}
Maya R. Gupta, Yihua Chen
\newblock{\ Theory and Use of the EM Algorithm}.
\newblock Now Publishers Inc, 2011


\bibitem{Ber04}
Bickel and Docksum
\newblock{\ Mathematical Statistics}
\newblock A Chapman $\&$ Hall Book
\end{thebibliography}
\end{frame}


\end{document} 